% Define \then como ⇒
\newcommand{\then}{\Rightarrow}

% Define \grad{f}{x} como ∇f(x)
\newcommand{\grad}[2]{\nabla #1(#2)}

% Conjuntos numéricos
\newcommand{\reals}{\mathbb{R}}      % 
\newcommand{\naturals}{\mathbb{N}}   % 
\newcommand{\integers}{\mathbb{Z}}   % 
\newcommand{\rationals}{\mathbb{Q}}  % 
\newcommand{\complex}{\mathbb{C}}    % 

\newcommand{\inprod}[2]{\left\langle #1, #2 \right\rangle}

\newif\ifshowcomments   % Define la condición

\showcommentstrue      % <- Puedes cambiar esto a \showcommentsfalse en el main si quieres ocultarlos

\ifshowcomments
  \newcommand{\nota}[1]{\textcolor{red}{[Nota: #1]}}  % Se ve en rojo
\else
  \newcommand{\nota}[1]{}  % No muestra nada
\fi
