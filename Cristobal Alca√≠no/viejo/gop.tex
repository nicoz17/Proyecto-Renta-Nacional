$$ I_{G_t} + I_{M_t} + I_{B_t} \leq s $$
Definido como los inventarios y la capacidad de almacenamiento $\forall t \in 1...T$
\\
Definimos una variable auxiliar como $A_{Gt}$ que almacena en un turno una cantidad de genero.
$$ A_{Gt} = I_{Gt} + G_t \forall t \in T $$
Donde queremos decir que vamos a usar auxiliarmente el genero que tengamos guardados y el que decidamos producir (en kg).
A su vez lo mismo con la ropa en mal estado donde:
$$A_{Mt} = I_{Mt} + kr_m\forall t \in 1...T$$ donde decimos que a lo mas vamos a usar la ropa mala que tengamos en inventario junto con la que 
nos llegue.
\\ 
Finalmente para la ropa buena tenemos algo como:
$A_{Bt} = I_{Bt} + k r_b + B_t$ donde decimos que usaremos auxiliarmente la ropa buena que tengamos en inventario junto con la que llega 
junto con la que decidmos producir.
\\
Notemos que solo podemos producir hasta donde el genero nos permite por lo que:
$$ B_t \leq A_{Gt} \forall t \geq 0 $$
Donde ademas definimos $S_t$ como la cantidad de prendas que vamos a enviar donde a lo mas podremos enviar:
$S_t \leq \frac{A_{Bt}}{p}$ prendas.

\\
Donde finalmente tenemos que además solo podemos producir tantos kilos buenos y genero bueno como nos lo permitan las horas lo cual expresamos como:
$$B_t \tau_b + G_t \tau_g \leq (w_0 + K_t) \cdot h$$
\\
Finalmente actualizamos los inventarios para la siguiente iteracion como:
\begin{align}
    I_{Gt+1} = (A_Gt - Bt)\\
    I_{Mt+1} = (A_Mt - Gt)\\
    I_{Bt+1} = (A_Bt - St \cdot p)\\
\end{align}
Finalmente nuestra función objetivo debiese verse como:
\begin{align}
    \min \sum_{t \leq T} \Big( (I_{Gt} + I_{Mt} + I_{Bt}) \cdot a  + (w_0 + K_t)(h)(cc) + G_t \cdot g + B_t \cdot n + \min(d_t-S_t, 0)(cp) + (K_t) (c_t)  \Big)
\end{align}
Donde G_t, B_t, K_t, S_t seran variables.